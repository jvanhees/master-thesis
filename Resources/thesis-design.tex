\documentclass{acm_proc_article-sp}

\usepackage{todonotes}
\usepackage{fontspec}
\usepackage{makeidx}
\usepackage{rotating}


\usepackage[
backend=biber,
style= numeric,
sorting=none
]{biblatex}
%\addbibresource{papers.bib}
 
\setmonofont{DejaVu Sans Mono}
 
\pagenumbering{arabic}

\begin{document}

\title{Thesis Design}
\subtitle{Master Thesis Information Studies}

\author{
Jorick van Hees \\
\texttt{\textbf{Blue Billywig}} \\
\texttt{jorick.vanhees@student.uva.nl} \\
\texttt{UvA Student Nr.: 10894020} \\
\texttt{VU Student Nr.: 2567527}
}

\maketitle

\category{Computing methodologies}{Computer vision}{Video summarization}

\section{Introduction}

With the introduction of high speed internet and the introduction of video sharing platforms, the use of online video increased dramatically and is one of the main media content formats used on the web today, with increasing popularity.

Newspaper websites contain dozens of pages that feature single articles, some of which might include video. A user navigates to these single article pages from overview pages, examples of which are the frontpage, category pages and archive listings. The overview pages often enrich the links to the article pages with a title, a small description and a visual preview.

The preview is often a small image (often not bigger than a few centimeters in size), related to the content, that should engage the user to click the link to view the article on the article page. This image is what we will call the `video thumbnail': A visual excerpt of the full video that is used in overview pages to let the user explore more content without viewing each page and video individually.

\section{Problem statement}

Thumbnails have always been images, in many instances a screen capture taken on a certain point of time from the video. Online content management systems (or video management systems) offer functionality to specify what frame is extracted from the video, or allow the content editor to upload an entirely new image to use as a thumbnail. This allows the editor to select a thumbnail that is more engaging to the user and will generate more views on the video.

There are several aspects in thi


\section{Domain}

This falls into the computer vision domain: The analysation of video content, extraction of features and 

\section{Research Question}




\printbibliography

\balancecolumns
\end{document}
