\documentclass{../resources/acm_proc_article-sp}

\usepackage{todonotes}
\usepackage{fontspec}
\usepackage{makeidx}
\usepackage{rotating}


\usepackage[
backend=biber,
style=numeric,
sorting=none
]{biblatex}
\addbibresource{../resources/lib.bib}
 
\setmonofont{DejaVu Sans Mono}
 
\pagenumbering{arabic}

\begin{document}

\title{Thesis Design}
\subtitle{Master Thesis Information Studies}

\author{
Jorick van Hees \\
\texttt{\textbf{Blue Billywig}}\titlenote{Blue Billywig, Catharina van Renneslaan 20, 1217 CX Hilversum, The Netherlands} \\
\texttt{jorick.vanhees@student.uva.nl} \\
\texttt{UvA Student Nr.: 10894020} \\
\texttt{VU Student Nr.: 2567527}
}

\maketitle

\category{Computing methodologies}{Computer vision}{Video summarisation}

\section{Introduction}

With the introduction of high speed internet and the introduction of video sharing platforms, the use of online video increased dramatically and is one of the main media content formats used on the web today, with increasing popularity in news media, business matters, education and the social web. According to \textcite{Cisco:2015wm}, 64 percent of the consumer internet traffic in 2014 was online video.

News media website navigation often exists of article pages and overview pages like the front page, designed to allow easy navigation to the article pages. The overview pages link to the article using a title, a small description and a visual preview. The visual preview is a small visual representation, or \textit{thumbnail}, of the content, that is only a few centimeters in size. The thumbnail should engage the user to follow the link to view the full article and is often the only visual reference.

Thumbnails have always been a strong visual cue for the user to find relevant content. A study from \textcite{Dziadosz:2002cl} in 2002 regarding thumbnails in their search engine results, showed that users made more accurate decisions when a thumbnail of the result was presented next to the textual result. In general, images tend to pull the audience to the content; it's what newspapers are doing for years with the big frontpage images. The contents of these thumbnails exist of stock photos, actual photos from the event, or other designed images containing company logos, previews of graphs and other relevant visuals. Overview pages linking to multiple news articles often use these thumbnails to improve the visual appearance of the website, but also try to tempt the user into viewing the article. These images have always been static, but with the increase of broadband internet speeds and reliability (even on mobile devices), opportunities arise to create moving thumbnails. These thumbnails may improve the user experience and could increase the chances of users watching the full article.

\begin{figure}[h]
  \includegraphics[width=\linewidth]{images/belfast-telegraph-thumbnail}
  \caption{Screencapture of the Belfast Telegraph frontpage (http://www.belfasttelegraph.co.uk/) on the 5th of february, showing a thumbnail.}
\end{figure}


\section{The moving thumbnail}

With the rise of video as a way of communication and distributing news content, new opportunities arise to attract users to specific content. Instead of using a static frame from the video, it is possible to extract a small portion of the video and use that as a \textit{moving thumbnail}. The moving thumbnail is a small video in both dimensions and length, that acts as a substitution for a regular thumbnail. In every way, the moving thumbnail is the same as a regular static thumbnail with the exception that it contains `moving images'.

Static thumbnails have different purposes in news media websites in relationship to the content: it can try to accurately describe the content, or it can attempt to seduce a user to view the full content (which is what we can call a teaser). The chosen role of the thumbnail depends heavily on the type of content, the strategic goals of the media company and the location of where the thumbnail is used. Since the moving thumbnail replaces the static thumbnail, it should serve the same purpose.

To pick the best (static) thumbnail for a desired goal, a number of systems exist to aid an editor into selecting a thumbnail. The most simple system features a time selector and a preview window in which the editor is able to select a frame on a certain time to use as thumbnail. Other systems use different video analysation techniques to propose a set of suitable options to the editor. In order to use a moving thumbnail as a replacement for the static thumbnail, the creation of these moving thumbnails should be fast and convenient.

\subsection{New challenges for moving thumbnails}

Thumbnails are not the main focus of the page, and will not be started by a user action like a regular video would by clicking a play button. The users attention might also not directly be focussed on the moving thumbnail, and sometimes even multiple moving thumbnails may show on a page. These issues are also present in static thumbnails, but since they have no dimension of time, this does not present itself as an issue. The workings and design of a moving thumbnail should account for this lack of initial focus.

A static thumbnail will show only a single frame of the original video, revealing little of the actual content. A moving thumbnail however, might already show the punch line of the video, eliminating the need for a user to click through to the full video. For end-users, this might not impose as a problem, however media companies gain their revenues by serving advertisements before a video, or next to an article. This highlights the need to create moving thumbnails that tease the user, instead of showing a summary.

A more technical challenge is the design of the moving thumbnail in relation to processing power, internet speeds and device limitations. Even small videos can be significantly larger in file size than static thumbnails, and may consume more processing power, especially on mobile devices. The technical details of the moving thumbnail should consider these limitations.

\subsection{Automatic generation of moving thumbnails}

Creating moving thumbnails manually for every video can be a time consuming task, even when presented with various editing tools that allow quick trimming. In order to solve this problem,  automatic generation of moving thumbnails is essential for the use and implementation of this new concept.

The system that automatically generates moving thumbnails from a video has to take several requirements into account. Examples of these requirements are that the video should be suited for small dimensions or that the user is invited to watch the whole video. These specific requirements should be defined in a later state, based on existing previous research in the field of static thumbnail generation, video navigation and video summarisation.

\section{Problem statement}
\label{sec:problem statement}

As of today, there is no system that generates moving thumbnails with the goal to tease the user to view the full video. The specifications and requirements for a moving thumbnail are unclear, a method to measure the effectiveness of the output of the system does not exist for this particular case, and since the concept is new, there is no data available to accurately train a system for this purpose.

The absence of data leads the core of this research: The goal is to build a system that generates moving thumbnails and evaluates the results automatically to validate the output and allows for rapid improvements without user testing. The automatic evaluation can not rely on real world data (since it does not exist yet) and functions as some sort of sanity check. When the system scores acceptable on the automatic evaluation, the results are implemented in an end user test scenario to evaluate the system and gain real world data for the purpose of generating moving thumbnails.

The research question is as follows:

\textbf{How can we build a system that automatically generates moving thumbnails that invite the user to view the full video?}

The following subquestions will support the main research questions:

\textit{How can the automatic thumbnail generator be automatically evaluated?}

\textit{What features play a role in the automatic generation of moving thumbnails?}

The user testing that follows allows us to check wether we have actually succeeded in building such a system. This results in the following subquestion:

\textit{How does the output of the system behave in a real world scenario?}

By answering these questions, we can describe a system that automatically generates moving thumbnails and evaluates its results, which are validated by a real world survey.

\subsection{Context}

The origin of the new concept of moving thumbnails lies at media companies, who constantly try to seek new ways to improve their website and the way users navigate their news. At Blue Billywig, we received a request to offer a way to show moving thumbnails, created manually by the editors. After investigating this new opportunity, it was decided that an automatic way of generating these moving thumbnails was required.

Blue Billywig is a company that builds and maintains a video management systems, in which its customers can upload videos, edit metadata, adjust the way these videos are displayed and publish the videos on their website. In a lot of ways, the systems works the same way as the populair video site YouTube does \cite{YouTube:A-l2msAp}, only focussed on the use by companies and without a community.

\section{Approach}

As mentioned in section \ref{sec:problem statement}, the research can be divided into three closely related subjects. The first subject is about the moving thumbnail generator itself, a system that uses various features to determine what to use from an input video to create a moving thumbnail as output. The second subject is about a validator that automatically validates the output of the system, to check the integrity of the generated moving thumbnails. The third subject is about the user test scenario that tests the output of the generator with actual users, which results can in turn be used to compare the validator results with the actual user testing, and show real world effectiveness of the thumbnail generator. This workflow is shown in figure \ref{fig:workflow}. These tree subjects combined form a study that covers all grounds in this new concept of automatically generating thumbnails.

\begin{figure}[h]
  \label{fig:workflow}
  \includegraphics[width=\linewidth]{images/thesis-system.pdf}
  \caption{Schematic workflow overview of the complete system, showing feedback directions on the generator and evaluation.}
\end{figure}

The dataset that will be used will be chosen on the availability of the data, the domain that the research aims for (news media) and the amount of metadata that is needed to gain accurate results. Blue Billywig has a number of customers that have data related to the research, properly annotated with relevant metadata and in satisfying quantity. Another advantage of using an `in-house' dataset is the accessibility of the data and video files. Blue Billywig encodes every video file to a number of different formats to allow playback on any device and internet connection. This preparation of data might significantly increase the progress of the study.

\section{Literature study}

Since the generation of moving thumbnails is a new concept, there is no directly related literature. However, it relates close to a number of other domains: Video navigation and interfaces, static thumbnail generation and video summarisation.

\subsection{Video navigation}

There are a lot of interfaces designed to assist humans in navigating a video library. A summary by \textcite{Schoeffmann:2010iw} describes over 40 different interfaces that use different techniques to allow convenient browsing through a collection of videos. These range from displaying a key frame best describing the video, or a collection of keyframes with their sizes related to the importance of the frame. Most of these interfaces have some sort of system that automatically determines what to display on the screen, taken into consideration the different conditions in which the system is used.

A study by \textcite{Hurst:2011jx} describes a user study to the recognition of video using different thumbnail sizes, numbers and various movement in the thumbnails. The study shows that users are able to handle multiple small thumbnails on mobile devices, especially when the thumbnails included motion.

Since one of the goals of the moving thumbnail is to improve the navigation of users in a news media website, video navigation literature could especially prove useful in the design and implementation of moving thumbnails in overview pages.

\subsection{Static thumbnail generation}

The issue of video navigation using thumbnails has been an active topic of research. In \citeyear{Kim:2015co}, a study by \textcite{Kim:2015co} describes a system that automatically combines video frames to generate a thumbnail containing more information that a single frame. Another study describes thumbnail candidate selection using image quality evaluation \cite{Zhang:2014jg}. A combination of internal and external analysation of the video content to select thumbnails is used by a study by \textcite{Liu:2015ux}. The techniques and analysation methods used in these systems can possibly be of use when evaluating the generated moving thumbnails.

Many systems that generate thumbnails use a ranking of different frames to propose a suitable thumbnail \cite{Choi:2015gm,Zhang:2012eo,Gao:2009dx}. In static thumbnail generation, this ranking can be used to select the best thumbnail. However, in moving thumbnail generation (or other video navigation interfaces) the ranking can be used to create a composition of the video. This creates opportunities to generate moving thumbnails that consist of different shots from the original video.

\subsection{Video summarisation}

Video summarisation is the domain that is closest to the concept of the moving thumbnail, since a moving thumbnail is basically a video summarisation with a few major differences. The length of a video summary would depend on the contents of the actual video, an hour long video would need a longer summary to contain all concepts in the video than a summary for a minute long video. The moving thumbnail however, should ideally always be around the same size, since it only has to attract users to view the full video, instead of showing what the video is about.

Techniques for video summarisation can be divided into three different categories: Internal analysation, external analysation and a hybrid of the two. Internal analysation techniques use information that is gained from analysing the video itself, while external analysation uses the information available outside the video (metadata like title, description or tags). The hybrid techniques combine both internal and external analysation \cite{Money:2008fn}.

The knowledge gained from video summarisation can be used to generate and extract relevant parts from a video, and those in turn can be used to create moving thumbnails. The difference between video summarisations and moving thumbnails has to be highlighted in order to modify (or use techniques from) video summarisation techniques in order to use them for another purpose. Various frameworks and surveys on different techniques highlight the differences and effectiveness in domains and content \cite{Money:2008fn,Ajmal:2012hi}, which might be useful in the design of the moving thumbnail generator.

\section{Planning}

Development of the system will be planned in a scrum-like planning, implemented in the scrum planning of Blue Billywig. This means that the planning is divided into sprints of two weeks. The planning, as shown in figure \ref{}, describes a number of topics that should be covered in those sprints. The columns `BB' describes the sprint numbers as used in the Blue Billywig sprint planning.

Working on the master thesis document itself is done throughout the whole planning, and has no defined deadlines other than the draft versions in sprint 9 and 10. 

\subsection{Plan B}

One of the highest priorities in the system is the user study. Without the user study, the results of the system, as well as the effectiveness of the validator cannot be measured. Thus, it is important that the user study is ready in sprint 6 to gather enough data to gain significance. Sprint 5 is mostly used to improve the overal system, and can be changed when the workflow isn't properly set up in sprint 4. The overal planning can be extended with one sprint by removing one of the draft versions of the master thesis. However, this should be an extreme measure when the system has severe delays and the system can not be validated yet.

%It is important to note that thumbnails are not the main content of a page. They are used to assist the user in navigating a page and scanning content, and should be available to the user at all times, wether he just landed on the page, or is still viewing the page after a few minutes.
%
%
%The moving thumbnail has a start- and an endpoint. moving thumbnail, the video will loop infinitely to make sure the user will always see a moving thumbnail. In addition, the video will automatically start playing when the webpage is loaded.
%
%Online content management systems (or video management systems) offer functionality to specify what frame is extracted from the video, or allow the content editor to upload an entirely new image to use as a thumbnail. This allows the editor to select a thumbnail that represents the video in the best way, or may attract most viewers.
%
%
%
%Unlike static image thumbnails, the moving thumbnail videos have a set length and will stop when the end is reached. In order to create a continues moving thumbnail, the video will loop infinitely to make sure the user will always see a moving thumbnail. In addition, the video will automatically start playing when the webpage is loaded and will contain no audio. The latter is done to avoid sudden unexpected sounds that may annoy the user when they visit the page, and to allow multiple moving thumbnails playing at the same time. 

%\section{Problem statement}
%
%\subsection{Requirements of automatic generation}
%\label{sec: requirements of automatic generation}
%
%The goal of the moving thumbnail is to attract users to follow the underlying link to the full article. In order to do this, the moving thumbnail has a few requirements that should be met when generating it.
%
%\begin{enumerate}
%  \item It should invite the user to watch the full video, using interesting shots that may rise the interest of the viewer.
%  \item The shot length and general timing of the moving thumbnail should be suited for viewing by a human being.
%  \item The shots used should be suited for the small dimensions of the moving thumbnail.
%  \item The moving thumbnail doesn't contain any audio, thus the contents should be interesting without audio.
%\end{enumerate}
%
%\subsection{Specification of the moving thumbnail}
%
%The moving thumbnail is not yet being used in practice and thus has no standard in terms of length, size or any other specification. In order to generate usable thumbnails, there are certain specifications that should be met. These specifications are part of the research.
%
%\begin{enumerate}
%	\item The length of the thumbnail.
%	\item The dimensions of the thumbnail. It should be noted that the actual display of the thumbnail in a production website might differ from these dimensions. The dimensions should provide a standard in the same manner as `720p' or `1080p' do in current online video usage.
%	\item A target file size of the generated thumbnail. This is important since video quality might degrade when using heavy compression in order to save bandwidth, which in turn might decrease the clarity of the moving thumbnail.
%\end{enumerate}

%\subsection{Research question}
%
%The knowledge gained from the study should be the specified features that are used in the generation of these video thumbnails, that are known to affect the effectiveness of the moving thumbnail. These effects can range from shot and overal length, to color saturation or facial recognition.
%
%The research question can be formulated as follows:
%
%\textbf{What features used in the generation of moving thumbnails affect the curiosity of the viewer, increasing the need for them to see the full video?}
%
%\section{Generating interesting moving thumbnails}
%
%To fulfil the requirements stated in section \ref{sec: requirements of automatic generation}, the main body of the thesis will consist of research to the appropriate features in the generation of moving thumbnails. These features are used to create a moving thumbnail that is interesting for the audience.

%\subsection{The dataset}
%
%The first step in this process is gathering a test and training dataset on which we can test the effect of the features. This dataset should contain both good examples (as in: interesting moving thumbnails) and bad examples (meaning non interesting moving thumbnails). Since we want to test multiple features, the number of videos in this dataset should be among the 100s to 1000s.
%
%Since the moving thumbnail is a new concept that is not actually implemented in a real-world scenario, we cannot get a dataset specifically for this goal. This means that we have to create a new dataset, which is possible using a number of methods.
%
%A specifically designed survey, which gathers data in a controlled environment using specifically selected videos. In order to gain accurate results, the number of people taking this survey would be too big to be realistic.
%
%A real world implementation on an existing website that gathers data using user analytics would be another way of creating a dataset. However, the results could be affected by factors outside the test environment, thus wouldn't be reliable. Finding a website that meets the requirements would be a challenge on its own.
%
%It might be possible to use an existing dataset that may not directly contain the metrics that are required, but can display significant changes when features are changed in other metrics that the dataset contains. An example of such a dataset is the `Vine' dataset, containing `Vine' videos that have a lot of views, or no views. The amount of views might specify wether a video is interesting or not. In this case, it would be very important to support the dataset with proper research and logic.
%
%\subsection{Research analysis}
%
%In order to measure the effectiveness of the system, the automatically generated moving thumbnails will be compared to static thumbnails. This will be done using a custom test setup (perhaps with A/B testing scenarios) in an isolated environment. If possible within the time constraints of the master thesis, there will be a real-world test with cooperation of a customer of Blue Billywig that is interested in using the technology when its ready for production.
%
%An analysis of the test results will determine of the moving thumbnail generation system has added value in terms of conversion and if the moving thumbnail is successful in attracting more audience to the article.
%
%\section{Dataset}
%
%Since the system has to work with existing data and results with which it can improve itself, the dataset is a very important part of the research and has a number of requirements that have to be met.
%
%\subsection{Size}
%
%The dataset should contain around 500 to 1000 videos. The system that will evaluate the videos on their ability to function as a teaser, requires learning data. The amount of features that help the system decide this will probably more than a few, thus creating a high-dimensional space in which the system has to find a pattern, increasing the need for a lot of data.
%
%\subsection{Training data}
%
%Since the system has to learn wether certain videos (or video fragments) are fit to function as a teaser, it requires known training data. The training data is a very key component in the dataset, since it will eventually decide what features in the video affect its popularity. First, this means that training data should be available at all, for example in the form of number of views on the video.
%
%Secondly, this data has to be reliable. Videos on YouTube are often getting watched since the uploader has a lot of 'subscribers', the custom video thumbnail might be very inviting or because the title contains effective keywords. All these factors have nothing to do with the actual quality of the video, and might result in false training data. It is therefor important that the training data is justified and that the use of it is supported with solid arguments.



\printbibliography

\balancecolumns
\end{document}
