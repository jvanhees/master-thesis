\documentclass{../resources/acm_proc_article-sp}

\usepackage{todonotes}
\usepackage{fontspec}
\usepackage{makeidx}
\usepackage{rotating}

\usepackage{booktabs}
\usepackage{multirow}

\usepackage[
backend=biber,
style=numeric,
sorting=none
]{biblatex}
%\addbibresource{/Volumes/Papers Library/Library/Collections/Video thumbnails/Ω Export/Video thumbnails.bib}
 
\setmonofont{DejaVu Sans Mono}
 
\pagenumbering{arabic}

\begin{document}

\title{Generating video thumbnails, a replacement for video preview images}
\subtitle{Master Thesis Information Studies}

\author{
Jorick van Hees \\
\texttt{\textbf{Blue Billywig}}\titlenote{Blue Billywig, Catharina van Renneslaan 20, 1217 CX Hilversum, The Netherlands} \\
\texttt{jorick.vanhees@student.uva.nl} \\
\texttt{UvA Student Nr.: 10894020} \\
\texttt{VU Student Nr.: 2567527}
}

\maketitle

\category{Computing methodologies}{Computer vision}{Video thumbnails}

\section{Introduction}

Thumbnail images for videos are used all over the web. They are small, static representations of videos, and provide a visual preview of the video itself. In combination with a title and description, they form one of the most common interfaces when dealing with a collection of videos. They increase the accuracy when conducting searches in video databases, improve the aesthetics of an overview page, and can increase engagement when using appealing thumbnails.

News media websites often have an overview page (like a front page) that display links to articles containing a video. These links are often presented using the title, a textual description and a thumbnail from the video. This thumbnail is often the only visual reference to the video, and is used to engage the user and improve the `click-through ratio'.

With the increase of broadband internet speeds and reliability (even on mobile devices), opportunities arise to use short videos as thumbnails. These video thumbnails could increase engagement and user experience compared to static thumbnails.

In order to use video thumbnails in current production workflows as a replacement for static thumbnails, a similar system has to be designed for video thumbnails. In this work, we will propose a way to automatically generate a selection of video thumbnail candidates, based on event detection and concept recognition. The system is then evaluated with a user study, in which the engagement of a video thumbnail is tested against a static thumbnail variant.

%todo Specifically include research question and subquestions?

%todo Do I need to explain this in the introduction?
Manually analysing a video in order to find a proper thumbnail is a time consuming task which an editor doesn't want to be bothered (todo: right word?) with. This is especially true when multiple videos are published each hour. To improve workflow and convenience when selecting a thumbnail, a machine can propose multiple candidates to the editor. This will save time and increases convenience. Such a workflow currently exists for static thumbnails using various algorithms.


\subsection{Related work - TODO}

Work in the field of video summarisation has a lot of common ground with the generation of video thumbnails. Both use similar data to extract the desired information from a video and its metadata, the same techniques in computer vision is used to process the data and the end result could be very similar. 

One could argue that the results from video summarisation could be used in some implementations of the video thumbnails. However, the use cases in both domains are vastly different in terms of user engagement. In general, a summary tries to accurately describe the contents of the video, which eliminates the need to view the full video. In turn, the goal of the thumbnail is to engage the user to view the full video. It tries to show just enough to trigger the user to view the remaining content. The vastly different goal of the video thumbnail has such an impact, that the generation of video thumbnails deserves its own separate task.


\subsection{Event detection}

Existing systems for video event detection based on the TRECVID Media Event Detection Task often use the provided training data. In our system, no training data is available. However, we can use a number of techniques in order to extract the most characteristic frames from the video.

\section{Dataset}



\section{Generating video thumbnails}
\label{system description}
% Todo good or bad
The main body of this paper consists of the design of a system that is able to generate video thumbnails, based on metadata, an editor-selected static thumbnails and the video itself. The system can be divided into four major stages: Candidate selection, candidate evaluation and topic-based model training. In the first stage, a number of moments in the video are selected and labeled as candidates. In the second stage, candidates are labeled with a ground truth value (?) based on their similarity to the editor-selected thumbnail. In the third stage, we cluster all videos in the dataset based on topics derived from their metadata. In the fourth and final stage, an SVM is trained for each cluster using the labeled candidates from each video in that cluster.



\subsection{Candidate selection}

Instead of evaluating all possible frames in a video using a sliding window, a number of candidates are selected from the video. The selection algorithm is inspired by the bag-of-fragments as described in \cite{Mettes:2015vg}. %todo More intro

Features from the video are extracted from individual frames with 1 frame per second using a convolutional neural network, trained on the ImageNet dataset \cite{Krizhevsky:2012wl} using CaffeNet \cite{Jia:2014cm}. The result is a high-level sparse feature representation of the frame with 1000 dimensions. This representation is used throughout the system as a representation of the frame. This method of extracting features proves to be very effective compared to low-level feature extraction for event-detection \cite{Habibian:2013ks, Althoff:2012gf} and object recognition. % Todo: References to object detection.

% Todo better explanation of slinding window
A comprehensive list of video thumbnail candidates is generated using a brute-force sliding window method. The video frames in the window are max-pooled into a vector representation to catch every concept in the video thumbnail. In order to prune the list of candidates, the video thumbnail is compared to a bag-of-fragments representation of the full video. The bag-of-fragments are created with the features extracted from all frames, clustered using K-means. The number K is calculated using the length of the video $K = \frac{frames}{20 * fps}$. The similarity between the video thumbnail and each bag-of-fragments is measured using cosine similarity. The resulting similarity vector (with K dimensions) is then normalised, to prioritise the most representing video thumbnail for each bag-of-fragments. Finally, a ranking is made for each bag-of-fragments where the top result in each ranking is selected as thumbnail candidate.

\subsection{Candidate evaluation}



\subsection{Topic clustering}

\subsection{Model training}

\section{User study}

% Todo include numbers from study
The video thumbnails generated by the system described in \ref{system description} have been tested in an A/B user survey against a baseline in the form of static thumbnails: 50\% of the respondents received the survey which included video thumbnails, while the other 50\% received a version with static thumbnails. The survey was conducted via a custom build website to ensure compatibility across multiple devices.

\subsection{Survey setup}

For each video, two previews were shown in successive order: The first preview contained only a title and description, while the second preview contained a title, description and a (static or video) thumbnail. An example of the latter preview is shown in figure... % Todo include figure with preview
After each preview, two statements were made about the video preview to measure the engagement of the participant towards the video, and wether the participant felt informed about the contents of the video:

\begin{itemize}
	\item I am interested in viewing the video (engagement).
	\item I know what to expect from the video (informative).
\end{itemize}

By comparing the difference in answers between the version with thumbnail and without thumbnail, we are able to measure the impact of using a thumbnail in the preview. This difference can then be compared between the static thumbnail and video thumbnail, allowing us to analyse the effect of a video thumbnail. This way, any preconception from the user about certain topics or videos can be taken into account.

The videos used in this survey were manually picked from the dataset based on number of views. Early feedback on the survey setup revealed that randomly selected videos would be uninteresting, regardless of the form of the video preview. The difference in target audience between the dataset source and survey respondents would be the primary explanation. The age of most of the videos is a second, since most of the news videos are outdated at the point of conducting the survey.

\subsection{Responses}

A total of 40 respondents participated, of which 20 received the version with video thumbnails, and 20 received the version with static thumbnails. Each participant received previews for a total of three videos, resulting in 60 responses for the static thumbnail, and 60 responses for the video thumbnail.


\end{document}







